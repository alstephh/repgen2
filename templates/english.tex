\BLOCK{ include "preamble.tex" }

\begin{document}

% header and footer
\pagestyle{fancy}
\fancyhf{} % clear existing header/footer entries

\begin{titlepage}

  \vspace*{\fill}

  \begin{center}
    \makebox[\textwidth]{\includegraphics[width=\paperwidth]{templates/title}}
  \end{center}

  \vfill
  
  {
    \Huge \textbf{\MakeUppercase{Penetration Test Report}}
    \vspace{1ex}
  }

  {
    \Large \textbf{\VAR{ project.customer }}
    %\vspace{1ex}
  }

  {
    \Large \textbf{\VAR{ project.name }}
    %\vspace{1ex}
  }

  \VAR{ project.report.revision }
\end{titlepage}

\fancyhead[L]{\VAR{ project.name }}
\fancyhead[R]{Confidential}

\fancyfoot[C]{\thepage}

{
  \vspace*{\fill}

  This report is for the sole information and use of \VAR{ project.customer|trim('.') }.

  \textbf{Secure By Default, Inc.} \\
  \href{tel:+0123456789}{+0 123 456 789} \\
  \href{mailto:contact@sbd.local}{contact@sbd.local} \\
  \href{https://sbd.local/}{https://sbd.local/}
}

\clearpage
\section*{Executive Summary}
%\addcontentsline{toc}{section}{Executive Summary}

\VAR{ project.customer }\ (i.e.\ the customer) engaged Secure By Default, Inc.\ to conduct a penetration test on their systems.
In accordance with the customer the following types of penetration tests have been conducted:

\begin{itemize}
  \BLOCK{ for service in project.services }
    \item \VAR{ service|markdown2latex }
  \BLOCK{ endfor }
\end{itemize}



%+++++++++++++++++++++JUST WITH CVSS++++++++++++++++++++++++++++++++++++++++++++
\BLOCK{ if "cvss" in issues[0]["severity"] }
  Vulnerabilities with severity scores between \VAR{ (issues|min(attribute='severity.number')).severity.number } and \VAR{ (issues|max(attribute='severity.number')).severity.number } (on a scale from 0--10) have been found (see Table~\ref{tab:vulnerabilities}).
  A detailed list thereof, including descriptions on how they were found, and recommendations for mitigations, can be found in Section~\ref{sec:results}.

  \begin{table}[h!]
    \centering
    \caption{The vulnerability classes, the highest severity, and the number of issues per class.}
    \label{tab:vulnerabilities}
    \begin{tabular}{lcr}
      \textbf{Class} & \textbf{Severity} & \textbf{Issues} \\
      \hline
      \BLOCK{ for class, items in issues|groupby("class") }
        \VAR{ class|title } & \minmalseveritygaugeH[0.2]{\VAR{ (items|max(attribute='severity.number')).severity.number }} & \VAR{ items|length } \\
      \BLOCK{ endfor }
      \hline
      \textbf{Total} & ~ & \textbf{\VAR{ issues|length }}
    \end{tabular}
  \end{table}
%+++++++++++++++++++++JUST WITH CVSS++++++++++++++++++++++++++++++++++++++++++++





%+++++++++++++++++++++JUST WITH SSVC (DONE!)++++++++++++++++++++++++++++++++++++++++++++
\BLOCK{ else }

 Vulnerabilities with severity scores between \VAR{ (issues|min(attribute='severity.value')).severity.category } and \VAR{ (issues|max(attribute='severity.value')).severity.category } (on a scale from Track--Act) have been found (see Table~\ref{tab:vulnerabilities}).
  A detailed list thereof, including descriptions on how they were found, and recommendations for mitigations, can be found in Section~\ref{sec:results}.


  \begin{table}[h!]
    \centering
    \caption{The vulnerability classes, the highest Decision, and the number of issues per class.}
    \label{tab:vulnerabilities}
    \begin{tabular}{lcr}
      \textbf{Class} & \textbf{Highest Decision} & \textbf{Issues} \\
      \hline
      \BLOCK{ for class, items in issues|groupby("class") }
        \VAR{ class|title } & \VAR{ (items|max(attribute='severity.value')).severity.category }} & \VAR{ items|length } \\
      \BLOCK{ endfor }
      \hline
      \textbf{Total} & ~ & \textbf{\VAR{ issues|length }}
    \end{tabular}
  \end{table}

\BLOCK{ endif }
%+++++++++++++++++++++JUST WITH SSVC (DONE!)++++++++++++++++++++++++++++++++++++++++++++


\VAR{ summary|markdown2latex }

\clearpage
\tableofcontents

\clearpage
\section{Introduction}

\VAR{ project.customer }\ (i.e.\ the customer) engaged Secure By Default, Inc.\ to conduct a penetration test on their systems.
In accordance with the customer the following types of penetration tests have been conducted:

\begin{itemize}
  \BLOCK{ for service in project.services }
    \item \VAR{ service|markdown2latex }
  \BLOCK{ endfor }
\end{itemize}

\subsection{Personnel}

The following people were involved in this penetration test:

\begin{itemize}
  \BLOCK{ for person in project.personnel }
    \item \VAR{ person|markdown2latex }
  \BLOCK{ endfor }
\end{itemize}

\subsection{Scope}

Between \VAR{ project.period.start } and \VAR{ project.period.end } the following components have been analyzed:

\begin{itemize}
  \BLOCK{ for scope in project.scope }
    \item \VAR{ scope|markdown2latex }
  \BLOCK{ endfor }
\end{itemize}

All test-related IP traffic originates from the following addresses:

\begin{itemize}
  \item \passthrough{192.168.0.0/24}
\end{itemize}

\subsection{Methodology}

As asked by Patrick Himler I would grab 5 vulnerabilities found in my (little) CTF experience. This
vulnerabilities would be managed as stand-alone so obviously the report sections are not connected to
each others and are part of different machine. I would use the template to make a brief description of every vulnerability context so you can make an idea of the rating I gave to them, since I am not that keen with SSVC you could expect unaccurate results. I will explain briefly why I choose a specific path in the Decision tree to make it clear also I will try to use all the capabilities provided by the tools (image, link, snippet of code).

\BLOCK{ if project.requirements }
  \subsection{Requirements}

  Secure By Default required the following from the customer:

  \begin{itemize}
    \BLOCK{ for req in project.requirements }
      \item \VAR{ req|markdown2latex }
    \BLOCK{ endfor }
  \end{itemize}
\BLOCK{ endif }

\BLOCK{ if project.provided_information }
  \subsection{Provided Information}

  The customer has provided the following additional information:

  \begin{itemize}
    \BLOCK{ for info in project.provided_information }
      \item \VAR{ info|markdown2latex }
    \BLOCK{ endfor }
  \end{itemize}
\BLOCK{ endif }

\BLOCK{ if limitations }
  \subsection{Limitations}

  Due to the time-constrained nature of audits, it is common to encounter coverage limitations.
  The following limitations were identified during this engagement.
  Secure By Default recommends further review and/or retesting of the affected systems/components.

  \VAR{ limitations|markdown2latex }
\BLOCK{ endif }

\subsection{Tools}

The following tools have been used during the engagement:

\VAR{ tools|markdown2latex }

\clearpage

\section{Results}
\label{sec:results}



        %+++++++++++++++++++++++++++ JUST CVSS ++++++++++++++++++++++
\BLOCK{ if "cvss" in issues[0]["severity"] }

  \BLOCK{ for group in groups|sort(attribute='order') }
    \begin{table}[h!]
      \centering
      \caption{Vulnerabilities and their severity\BLOCK{ if group.name } (component ``\VAR{ group.name }'')\BLOCK{ endif }.}
      \begin{tabular}{ p{0.65\linewidth} c d{1} }

        

        \textbf{Vulnerability} & \multicolumn{2}{c}{\textbf{Severity}} \\
        \hline
        \BLOCK{ for issue in issues|sort(attribute='severity.number', reverse=true) if issue.group.order == group.order }
          \hyperref[\VAR{ issue.label }]{\VAR{ issue.title }} & \minmalseveritygaugeH[0.2]{\VAR{ issue.severity.number }} & \VAR{ issue.severity.number } \\




        \BLOCK{ endfor }
      \end{tabular}
    \end{table}
  \BLOCK{ endfor }
\BLOCK{ endif }
        %+++++++++++++++++++++++++++ JUST CVSS ++++++++++++++++++++++








        %+++++++++++++++++++++++++++ JUST SSVC ++++++++++++++++++++++

\BLOCK{ if "ssvc" in issues[0]["severity"] }

  \BLOCK{ for group in groups|sort(attribute='order') }
      \begin{table}[h!]
        \centering
        \caption{Vulnerabilities and their severity\BLOCK{ if group.name } (component ``\VAR{ group.name }'')\BLOCK{ endif }.}
        \begin{tabular}{ p{0.65\linewidth} c d{1} }

          

          \textbf{Vulnerability} & \textbf{Decision} \\
          \hline
          \BLOCK{ for issue in issues|sort(attribute='severity.value', reverse=true) if issue.group.order == group.order }
            \hyperref[\VAR{ issue.label }]{\VAR{ issue.title }} & \VAR{ issue.severity.category } \\


          \BLOCK{ endfor }
        \end{tabular}
      \end{table}
    \BLOCK{ endfor }

\BLOCK{ endif }

        %+++++++++++++++++++++++++++ JUST SSVC ++++++++++++++++++++++


\clearpage

\BLOCK{ for group in groups|sort(attribute='order') }

  % loop over each of the group's issues ---------- SERIOUSLY I DON'T UNDERSTAND THIS WTF
  \BLOCK{ for issue in issues|sort(attribute='severity.number', reverse=true) if issue.group.order == group.order }
    \newpage
    
    \subsection{\VAR{ issue.title }}
    \label{\VAR{ issue.label }}

    \VAR{ issue.description|markdown2latex }

    \subsubsection{Evidence}

    \BLOCK{ for id, evidence in issue.evidence|items() }
      \paragraph{\VAR{ id }}

      \VAR{ evidence|markdown2latex }
    \BLOCK{ endfor }

    \subsubsection{Affected Assets}

    \VAR{ issue['affected assets']|markdown2latex }

    \subsubsection{Severity}

    \begin{center}
      
      %----ADJUST THIS FOR SSVC BUT CHOOSE THE METHOD FOR THE TABLE (WORKE BUT DOUBLE CHECK)----
      \BLOCK{ if "cvss" in issues[0]["severity"] }
        
      \severitygauge{\VAR{ issue.severity.number }}{\VAR{ issue.severity.number }}

        \BLOCK{ if issue.severity.cvss }
          \BLOCK{ if issue.severity.cvss[0] == '2'}
            \CVSStwo{\VAR{ issue.severity.cvss|join('}{') }}
          \BLOCK{ else }
            \CVSSthree{\VAR{ issue.severity.cvss|join('}{') }}
          \BLOCK{ endif }
        \BLOCK{ endif }

        \BLOCK{ if issue.severity.dread }
          \DREAD{\VAR{ issue.severity.dread|join('}{') }}  
        \BLOCK{ endif }
      
      \BLOCK{ else }
        \severitygaugeSSVC[2]{\VAR{issue.severity.value}} \newline

        \SSVCtwo{\VAR{ issue.severity.ssvc|join('}{') }}

      \BLOCK{ endif }  


    \end{center}
      % ---------------------------------------------------------------------------------



    \BLOCK{ if issue.recommendations }
      \subsubsection{Recommendations}

      \VAR{ issue.recommendations|markdown2latex }
    \BLOCK{ endif }

    \BLOCK{ if issue.references }
      \subsubsection{References}

      \VAR{ issue.references|markdown2latex }
    \BLOCK{ endif }

    \BLOCK{ if issue.images }
      \subsubsection{Images}
      
      \BLOCK{ for image in issue.images }
        \begin{minipage}[c]{\textwidth}
          \centering
          \includegraphics[width=\textwidth]{\VAR{ group.graphics_path }\VAR{ image.file }}
          \captionof{figure}{\VAR{ image.caption }}
          \label{\VAR{ issue.label }:\VAR{ image.file }}
          \vspace{4ex}
        \end{minipage}
      \BLOCK{ endfor }
    \BLOCK{ endif }

    \clearpage

  \BLOCK{ endfor }
  
\BLOCK{ endfor }

\end{document}
